% This is the Reed College LaTeX thesis template. Most of the work
% for the document class was done by Sam Noble (SN), as well as this
% template. Later comments etc. by Ben Salzberg (BTS). Additional
% restructuring and APA support by Jess Youngberg (JY).
% Your comments and suggestions are more than welcome; please email
% them to cus@reed.edu
%
% See http://web.reed.edu/cis/help/latex.html for help. There are a
% great bunch of help pages there, with notes on
% getting started, bibtex, etc. Go there and read it if you're not
% already familiar with LaTeX.
%
% Any line that starts with a percent symbol is a comment.
% They won't show up in the document, and are useful for notes
% to yourself and explaining commands.
% Commenting also removes a line from the document;
% very handy for troubleshooting problems. -BTS

% As far as I know, this follows the requirements laid out in
% the 2002-2003 Senior Handbook. Ask a librarian to check the
% document before binding. -SN

%%
%% Preamble
%%
% \documentclass{<something>} must begin each LaTeX document
\documentclass[12pt,twoside]{reedthesis}
% Packages are extensions to the basic LaTeX functions. Whatever you
% want to typeset, there is probably a package out there for it.
% Chemistry (chemtex), screenplays, you name it.
% Check out CTAN to see: http://www.ctan.org/
%%
\usepackage{graphicx,latexsym}
\usepackage{amssymb,amsthm,amsmath}
\usepackage{longtable,booktabs,setspace}
\usepackage{chemarr} %% Useful for one reaction arrow, useless if
% you're not a chem major
\usepackage[hyphens]{url}
\usepackage{rotating}
% \usepackage{natbib}
% Comment out the natbib line above and uncomment the following two
% lines to use the new
% biblatex-chicago style, for Chicago A. Also make some changes at
% the end where the
% bibliography is included.
\usepackage{biblatex-chicago}
\bibliography{thesis}

% \usepackage{times} % other fonts are available like times, bookman,
% charter, palatino

\title{Empirical Analysis of Fair Hierarchical Clustering Algorithms}
\author{Param Kapur}
% The month and year that you submit your FINAL draft TO THE LIBRARY
% (May or December)
\date{December 2025}
\division{Mathematical and Natural Sciences}
\advisor{Harper Knittel}
%If you have two advisors for some reason, you can use the following
%\altadvisor{Your Other Advisor}
%%% Remember to use the correct department!
\department{Computer Science}
% if you're writing a thesis in an interdisciplinary major,
% uncomment the line below and change the text as appropriate.
% check the Senior Handbook if unsure.
%\thedivisionof{The Established Interdisciplinary Committee for}
% if you want the approval page to say "Approved for the Committee",
% uncomment the next line
%\approvedforthe{Committee}

\setlength{\parskip}{0pt}
%%
%% End Preamble
%%
%% The fun begins:
\begin{document}

\maketitle
\frontmatter % this stuff will be roman-numbered
\pagestyle{empty} % this removes page numbers from the frontmatter

% Acknowledgements (Acceptable American spelling) are optional
% So are Acknowledgments (proper English spelling)
\chapter*{Acknowledgements}
I want to thank a few people.

% The preface is optional
% To remove it, comment it out or delete it.
\chapter*{Preface}
This is an example of a thesis setup to use the reed thesis document class.

\chapter*{List of Abbreviations}
You can always change the way your abbreviations are formatted. Play
around with it yourself, use tables, or come to CUS if you'd like to
change the way it looks. You can also completely remove this chapter
if you have no need for a list of abbreviations. Here is an example
of what this could look like:

\begin{table}[h]
  \centering % You could remove this to move table to the left
  \begin{tabular}{ll}
    \textbf{ABC}    &  American Broadcasting Company \\
    \textbf{CBS}    &  Columbia Broadcasting System\\
    \textbf{CDC}    &  Center for Disease Control \\
    \textbf{CIA}    &  Central Intelligence Agency\\
    \textbf{CLBR}   &  Center for Life Beyond Reed\\
    \textbf{CUS}    &  Computer User Services\\
    \textbf{FBI}    &  Federal Bureau of Investigation\\
    \textbf{NBC}    &  National Broadcasting Corporation\\
  \end{tabular}
\end{table}

\tableofcontents
% if you want a list of tables, optional
\listoftables
% if you want a list of figures, also optional
\listoffigures

% The abstract is not required if you're writing a creative thesis
% (but aren't they all?)
% If your abstract is longer than a page, there may be a formatting issue.
\chapter*{Abstract}
The preface pretty much says it all.

\chapter*{Dedication}
You can have a dedication here if you wish.

\mainmatter % here the regular arabic numbering starts
\pagestyle{fancyplain} % turns page numbering back on

%The \introduction command is provided as a convenience.
%if you want special chapter formatting, you'll probably want to
% avoid using it altogether

\chapter*{Introduction}
\addcontentsline{toc}{chapter}{Introduction}
\chaptermark{Introduction}
\markboth{Introduction}{Introduction}
% The three lines above are to make sure that the headers are right,
% that the intro gets included in the table of contents, and that it
% doesn't get numbered 1 so that chapter one is 1.

% Double spacing: if you want to double space, or one and a half
% space, uncomment one of the following lines. You can go back to
% single spacing with the \singlespacing command.
\onehalfspacing
% \doublespacing

Algorithmic development lies at the core of computer science, driving
innovations across virtually every technology and application domain.
Theoretical research provides essential insights, including
guarantees on correctness, computational complexity, and solution
quality. These theoretical properties guide the initial design and
selection of algorithms. Yet, translating these theoretical
guarantees into practical, real-world performance is often complex
and unpredictable. Real-world factors such as implementation
overhead, hardware-specific nuances, unique data characteristics, and
hidden constant factors can substantially alter an algorithm’s
observed performance compared to its theoretical expectations.

This disconnect—commonly referred to as the \emph{theory–practice
gap}—is particularly pronounced in areas that handle massive
datasets, require real-time or near-real-time responses, or involve
balancing multiple objectives such as fairness against traditional
performance metrics. Fair hierarchical clustering, for instance,
comes with theoretical bounds on the expected increase in clustering
cost due to fairness constraints; however, actual performance
trade-offs related to runtime and solution quality under realistic
conditions often remain unclear. Similarly, theoretical predictions
about parallel algorithms’ speedups can fail to materialize in
practice due to overlooked factors like communication overhead, while
approximation algorithms frequently outperform their theoretical
worst-case guarantees on typical problem instances.

Addressing this theory–practice gap is crucial because it influences
the adoption and practical impact of algorithms. While theory
provides essential worst-case or average-case assurances,
practitioners often lack reliable, evidence-based guidance to predict
real-world performance accurately. Consequently, this uncertainty can
either deter the use of theoretically superior algorithms or result
in suboptimal algorithmic choices based on anecdotal experiences or
incomplete testing. Despite its significance, systematic empirical
evaluation of algorithmic performance—especially in domains
exhibiting notable theory-practice discrepancies—is still rare.

This thesis tackles precisely this issue by providing a
comprehensive, structured empirical analysis designed to bridge the
existing theory–practice gap. Through an extensive literature review,
I first identify key algorithmic areas—such as fair hierarchical
clustering, parallel graph processing, and approximation methods—that
demonstrate significant divergence between theoretical predictions
and empirical observations. Recognizing the need for rigorous,
reproducible, and standardized testing, I have developed a robust
benchmarking framework specifically aimed at controlled
experimentation and comprehensive performance evaluations.

Utilizing this benchmarking environment, the research reported in
this thesis includes extensive empirical studies involving both
synthetic datasets—which probe specific theoretical
characteristics—and diverse real-world datasets that reflect typical
practical scenarios. These experiments systematically compare
theoretical metrics like runtime complexity, approximation ratios,
and fairness guarantees with empirical outcomes such as actual
execution times, memory consumption, solution quality, and fairness
metrics. Rather than isolated studies, this thesis presents an
integrated view, uncovering detailed patterns and insights into
algorithmic behavior under realistic conditions.

Ultimately, the contribution of this thesis lies in providing clear,
systematic empirical evidence to better inform algorithm selection
and implementation decisions, effectively bridging the gap between
theoretical expectations and practical realities.

\chapter{Background}
This is the first page of the first chapter. You may delete the
contents of this chapter so you can add your own text; it's just here
to show you some examples.

\section{Fairness Concepts in Machine Learning}\label{sec:fairness_concepts}
Machine learning has fundamentally transformed decision-making by
allowing algorithms to discover intricate and
meaningful patterns directly from data, without relying explicitly on
predefined rules. Rather than encoding
human expertise through manual programming, machine learning
algorithms generalize from examples. This inductive
process, which generalizes observed cases to unseen scenarios,
enables the algorithm to identify underlying patterns
from historical examples and predict future outcomes. However, such a
reliance on historical data inherently carries
risks, especially when data reflects existing societal biases,
stereotypes, or inequalities.

\subsection{Algorithmic Bias}\label{subsec:algorithmic_bias}

Algorithmic bias refers to systematic and repeatable errors or unfair
outcomes produced by machine learning models
due to biases embedded within the training data or algorithm design.
According to existing literature
\cite{barocas2016big,pessach2020algorithmic}, algorithmic bias
commonly originates from:

\begin{itemize}
  \item \textbf{Biases inherent in training datasets:} These biases
    result from biased human decisions, measurement errors,
    reporting inaccuracies, or historical prejudices embedded in
    datasets. Machine learning algorithms, aiming at optimizing
    prediction accuracy, often replicate these biases.
  \item \textbf{Biases due to missing data:} When datasets lack
    sufficient representation from certain groups or have
    significant data omissions, the resulting models fail to
    accurately represent the target population.
  \item \textbf{Algorithmic optimization bias:} Typical optimization
    objectives, such as minimizing aggregate prediction
    errors, tend to favor majority groups, often leading to poorer
    performance for minority groups.
  \item \textbf{Bias from proxy attributes:} Non-sensitive attributes
    may indirectly capture sensitive information
    (e.g., race, gender, age), unintentionally introducing biases
    even when sensitive attributes are explicitly excluded
    from the dataset.
\end{itemize}

\subsection{Defining Fairness in Machine
Learning}\label{subsec:fairness_definitions}

Given the increasing use of machine learning in high-stakes domains,
rigorous fairness definitions have emerged to guide
algorithmic development. These definitions typically fall into two
broad categories: individual fairness and group fairness.

\paragraph{Individual Fairness.}
Individual fairness requires models to produce similar outputs for
similar individuals, where similarity is assessed based
on relevant non-sensitive features. Formally, individual fairness can
be articulated using Lipschitz continuity
as follows \cite{dwork2012fairness}:
\[
  d(\text{output}(x), \text{output}(x')) \leq \rho \cdot d(x, x'),
\]
where \(x\) and \(x'\) are individuals with comparable non-sensitive
attributes, and \(\rho\) is a small constant. This definition
emphasizes the fair treatment of similar cases on an individual
basis, ensuring minimal unjustified variability.

\paragraph{Group Fairness.}
Group fairness demands that statistical outcomes of algorithms be
equitable across predefined demographic groups.
This approach explicitly acknowledges and attempts to rectify
societal disparities. Notable metrics
for group fairness include:

\begin{itemize}
  \item \textbf{Demographic Parity:} Ensures equal rates of positive
    predictions across demographic groups:
    \[
      P(R = 1 \mid A = a) = P(R = 1 \mid A = b),
    \]
    where \(R\) denotes the model's prediction and \(A\) represents a
    protected attribute (e.g., race, gender).

  \item \textbf{Equal Opportunity:} Requires equal true positive
    rates among groups, ensuring fairness in the
    allocation of positive outcomes given actual positives:
    \[
      P(R = 1 \mid Y = 1, A = a) = P(R = 1 \mid Y = 1, A = b),
    \]
    with \(Y\) representing the true outcome.

  \item \textbf{Equalized Odds:} Further requires equal true positive
    and false positive rates across groups,
    encompassing both success and error equity:
    \[
      P(R = 1 \mid Y = y, A = a) = P(R = 1 \mid Y = y, A = b), \quad
      \forall y \in \{0, 1\}.
    \]
\end{itemize}

Each of these metrics presents trade-offs, and no universal criterion
exists to satisfy all simultaneously,
leading to a fundamental tension explored later.

\subsection{Real-World Instances and Ethical Dimensions of
Algorithmic Bias}\label{subsec:real_world_bias}

Concrete examples vividly illustrate the risks associated with biased
algorithms. One widely cited example is the COMPAS algorithm,
frequently used in criminal justice for predicting recidivism.
Investigations revealed significant racial biases, incorrectly
labeling Black defendants as high-risk at nearly twice the rate of
White defendants who later re-offended~\cite{angwin2016machine}.
Similarly, facial recognition software has consistently demonstrated
higher error rates for darker-skinned individuals, exacerbating risks
of racial profiling and wrongful identification.

Algorithmic biases also permeate employment contexts, where
historical data reflecting past hiring decisions embed biases against
women or minority groups, perpetuating discrimination through
ostensibly neutral automated decision-making systems. When training
algorithms on historical employment records, biases and stereotypes
embedded in the data disproportionately disadvantage female candidates.

A particularly telling example emerges from machine translation.
Consider translating sentences from English to Turkish and then back
into English, as illustrated in Figures~\ref{fig:eng-to-turkish}
and~\ref{fig:turkish-to-eng}. Turkish pronouns are gender-neutral,
but when translated back into English, gender-specific pronouns are
inferred based on statistical associations. As a result, occupations
stereotypically associated with men—such as ``engineer'' or
``doctor''—are translated back using male pronouns, while occupations
stereotypically associated with women—such as ``nurse''—return female
pronouns. This phenomenon arises from two biases embedded in training
datasets: real-world labor market statistics reflecting historical
occupational distributions and the ``male-as-norm'' bias, whereby
male pronouns are preferentially selected when gender is ambiguous or
unknown~\cite{caliskan2017semantics}.

\begin{figure}[h]
  \centering
  \includegraphics[width=0.9\textwidth]{sections/background/turkish-to-eng.png}
  \includegraphics[width=0.9\textwidth]{sections/background/eng-to-turkish.png}
  \caption{Translations of gender-specific English sentences into
  gender-neutral Turkish and then back to English.}
  \label{fig:eng-to-turkish}
\end{figure}

Attempts to mitigate biases by removing explicitly sensitive
attributes (such as gender or race) from training datasets frequently
fall short due to \textit{proxy variables}. Proxy attributes—such as
the age at which individuals start programming—can inadvertently
encode sensitive information like gender, reinforcing biases even in
their absence. Additionally, biases due to disparities in sample
sizes among demographic groups lead to poorer model performance for
minority groups, reinforcing systematic inequities~\cite{barocas2016big}.

Beyond technical considerations, fairness intersects profoundly with
ethical and legal imperatives. Legal frameworks such as Title VII of
the U.S. Civil Rights Act mandate nondiscrimination in employment.
Similarly, regulatory initiatives like the European Union’s GDPR and
proposed AI Act embed fairness, transparency, and equity into
regulatory requirements for algorithmic systems.

Ethically, ensuring fairness aligns closely with broader principles
of justice and equity, particularly as algorithmic systems
increasingly influence societal outcomes such as employment,
education, and criminal justice. Effective fairness interventions
help prevent reinforcing historical injustices, foster social equity,
and maintain public trust in algorithm-driven decision-making.

\subsection{Challenges, Limitations, and Future
Directions}\label{subsec:challenges_future}
(TODO: citations here need to be filled in carefully I relied a bit
on yt tAjFuhkiV2c)

The growing emphasis on fairness in algorithmic decision-making,
particularly in clustering and machine learning contexts, has
significantly advanced our understanding of bias mitigation.
Nevertheless, this field continues to face considerable challenges
and inherent limitations.

One prominent difficulty arises from conflicting fairness criteria.
It is mathematically impossible to simultaneously satisfy all
fairness definitions—such as Demographic Parity, Equal Opportunity,
and Equalized Odds—highlighting the necessity for context-specific
fairness solutions. For example, attempts to apply broad fairness
concepts, initially developed within supervised learning frameworks,
to clustering tasks often encounter mismatches in meaning. Individual
fairness definitions emphasizing distributional equity may not
translate effectively into clustering contexts, where groups or
clusters often lack inherent meaning until assigned. This mismatch
underscores the need to carefully adapt fairness criteria from
supervised learning to unsupervised scenarios.

Moreover, algorithmic fairness interventions rarely function in
isolation. They typically constitute components within broader
socio-technical systems, necessitating careful consideration of both
upstream inputs and downstream impacts. The removal of sensitive
variables, a common fairness strategy, can inadvertently lead to
unintended consequences. For instance, the practice of "Ban the Box,"
intended to eliminate employment discrimination by prohibiting
questions about criminal history, inadvertently increased racial
discrimination as employers began using race as a proxy variable.
This underscores the complexity and potential pitfalls inherent in
algorithmic fairness interventions and highlights the importance of
anticipating and managing unintended downstream consequences.

Further complications emerge from the dynamic nature of real-world
data. In applications such as school districting or political
redistricting, clustering algorithms are applied to dynamic
populations, where individuals may relocate in response to
algorithmic interventions. Historical efforts like school busing
aimed at integrating racially segregated districts illustrate how
algorithmic clustering solutions can unintentionally disrupt
communities or exacerbate segregation through mechanisms such as
"white flight." This historical context reveals the critical need to
engage deeply with domain-specific constraints, legal frameworks, and
stakeholder needs, reinforcing that algorithmic solutions must be
cognizant of broader historical and societal dynamics.

Indeed, addressing these multidimensional challenges requires
interdisciplinary collaboration beyond computer science. Integrating
insights from fields such as sociology, criminology, economics, law,
and ethics is critical. For instance, fairness research often
overlooks valuable contributions from disciplines like criminology or
education policy, which provide nuanced understandings of systemic
inequities and practical constraints. Collaboration with experts from
these domains can guide the appropriate adaptation of algorithmic
methods to complex social contexts, ensuring solutions align closely
with practical realities and ethical standards.

Another essential consideration is stakeholder engagement. Too often,
fairness solutions are developed paternalistically, without
adequately involving those directly impacted. Engaging
stakeholders—including affected communities, policy experts, and
practitioners—in defining fairness and assessing interventions can
prevent misguided assumptions and ensure that algorithmic systems
genuinely serve intended beneficiaries. This is particularly evident
in sensitive domains such as criminal justice, healthcare, and
education, where the risk of inadvertently causing harm or
perpetuating injustices remains high.

Finally, reliance on commonly used benchmark datasets, such as the
COMPAS, Adult, and German Credit datasets, introduces risks of
replicating inherent data biases and inaccuracies. Issues such as
noisy demographic labels, inappropriate or misleading features, and
the misalignment of fairness labels highlight critical weaknesses in
the existing empirical evaluation landscape. Consequently, rigorous
methodological scrutiny and the development of better benchmarks
reflecting real-world complexities and accurate demographic
information are urgently needed.

In summary, future directions in algorithmic fairness research must
navigate inherent mathematical and practical complexities through
rigorous interdisciplinary collaboration, careful stakeholder
engagement, and meticulous empirical practices. By integrating
technical algorithmic approaches with broader societal, legal, and
ethical considerations, researchers can develop more robust,
equitable, and practically viable solutions to mitigate algorithmic
biases and their profound societal implications.


\section{Preliminaries and Fairness in Clustering}\label{sec:fair_clustering}
\subsection{Clustering Algorithms}
A clustering algorithm \(A\) partitions an input dataset \(X \in \mathbb{R}^{n \times m}\) into \(k\) clusters, where \(k \leq n\). Formally, the algorithm outputs a set \(C = \{C_1, C_2, \dots, C_k\}\), with each cluster \(C_i \subseteq X\), such that each data sample \(x \in X\) is assigned to at least one cluster. Depending on cluster assignment strategies, clustering is broadly categorized into two types:

\begin{itemize}
    \item \textbf{Hard clustering:} Each data point \(x\) belongs to exactly one cluster.
    \item \textbf{Soft clustering:} Data points may belong partially or probabilistically to multiple clusters.
\end{itemize}

In clustering tasks, unlike supervised learning, labels for data samples are unavailable. Consequently, the same dataset \(X\) is used for both training and evaluating the clustering outcomes. This complicates the definition and enforcement of fairness, as conventional fairness measures for supervised methods rely on labels to evaluate biases or discrimination.

The number of clusters \(k\) can either be provided as an input parameter or determined by the clustering method itself. For instance, in \(k\)-means, \(k\) is predefined, whereas hierarchical clustering outputs a dendrogram without a predetermined \(k\), allowing users to choose the number of clusters post hoc.

\subsection{Taxonomy of Clustering Methods}
A diverse set of clustering methodologies exists, each employing distinct strategies and assumptions. Following the taxonomy presented by Xu et al. \cite{XuSurvey}, we classify clustering algorithms as follows:

\begin{enumerate}

    \item \textbf{Center-Based Clustering:} These algorithms partition the data to minimize an error metric between samples and their cluster center. The canonical example is \(k\)-means, which minimizes the squared Euclidean distance.

    \item \textbf{Hierarchical Clustering:} This approach creates a binary tree (dendrogram) where the root node represents the entire dataset, leaf nodes represent individual samples, and intermediate nodes represent clusters. Hierarchical clustering methods are either agglomerative (bottom-up) or divisive (top-down).

    \item \textbf{Mixture Model-Based Clustering:} A probabilistic approach that assumes data points originate from a mixture of underlying distributions. Algorithms in this class, such as Gaussian Mixture Models (GMM), optimize parameters to best fit the data distribution, typically via Expectation Maximization.

    \item \textbf{Graph-Based Clustering:} Data is first represented as a graph, where vertices represent samples, and edges denote similarity or proximity between samples. Clustering is performed by partitioning the graph based on its spectral properties, commonly by using eigenvectors of the Laplacian matrix. Graph-based clustering is particularly advantageous in scenarios involving complex relational data or non-convex clusters.

    \item \textbf{Fuzzy Clustering:} Unlike traditional clustering, fuzzy clustering assigns samples a grade of membership to clusters. Fuzzy C-Means (FCM) is the archetypal algorithm, where data points have partial cluster memberships, reflecting uncertainty or gradual boundaries between clusters.

    \item \textbf{Combinatorial Search-Based Clustering:} Many clustering objectives are NP-hard, prompting approaches to reformulate the clustering problem as a combinatorial optimization task. Evolutionary algorithms, genetic algorithms, and simulated annealing are typical techniques used to explore solutions efficiently.

\end{enumerate}

\subsection{Fairness in Machine Learning}
Fairness in machine learning models can be ensured at different stages of the learning pipeline, typically categorized into:

\begin{enumerate}
    \item \textbf{Pre-processing}: The original dataset is modified prior to clustering to ensure fair representation or remove biases.
    \item \textbf{In-processing}: Fairness constraints are directly integrated into the clustering algorithm.
    \item \textbf{Post-processing}: The output clusters undergo modification to meet fairness constraints, which is uncommon for clustering due to the absence of a separate validation set.
\end{enumerate}

Since clustering is inherently unsupervised, without labels to quantify fairness explicitly, defining and enforcing fairness becomes notably challenging. Fairness in clustering often focuses on ensuring adequate representation of protected groups within each cluster (group-level fairness) or ensuring similar individuals are clustered similarly (individual-level fairness).

\subsection{Fairness Notions for Clustering}
Fairness notions in clustering can be classified into several categories:

\begin{itemize}
    \item \textbf{Group-Level Fairness}: Inspired by the Disparate Impact doctrine, these notions aim to ensure no protected group (e.g., based on ethnicity or gender) is disproportionately disadvantaged. For example, the \emph{balance} measure quantifies how proportionally protected groups are represented in clusters.

    \item \textbf{Individual-Level Fairness}: Ensures similar individuals receive similar clustering outcomes, typically defined using a dissimilarity metric.

    \item \textbf{Algorithm Agnostic Notions}: Defined independently of the clustering algorithm and applicable universally across clustering methods.

    \item \textbf{Algorithm Specific Notions}: Tailored specifically for certain clustering objectives or algorithms, such as the social fairness cost for \(k\)-means clustering.
\end{itemize}

Prominent fairness notions include:
\begin{itemize}
    \item \textbf{Balance:} Ensures proportional representation of protected groups in clusters.
    \item \textbf{Bounded Representation:} Enforces constraints on the maximum and minimum proportions of protected groups allowed in clusters.
    \item \textbf{Max Fairness Cost:} Measures deviations from an ideal proportional representation.
    \item \textbf{Social Fairness Cost:} Focuses on equitable representation of protected groups within the clustering cost objective.
\end{itemize}

Balancing fairness constraints with clustering quality often presents trade-offs, motivating methods that aim to find \emph{Pareto-optimal} solutions that best balance these competing objectives \cite{ChhabraOverview}.


\section{Hierarchical Clustering and Cost}
\subsection{Hierarchical Clustering: Definition and Approaches}\label{subsec:hierarchical_clustering}

Given the ethical, societal, and practical implications of algorithmic decision-making discussed previously, understanding the methodologies that underpin these algorithms becomes essential. Among these methodologies, \textit{hierarchical clustering} stands out as an influential unsupervised learning technique widely applied across numerous domains to identify natural groupings and reveal underlying structures within complex data.

Hierarchical clustering organizes a dataset \(X\) into a structured hierarchy of nested clusters, represented by a tree-like diagram known as a \textit{dendrogram}. Each leaf node corresponds to an individual data point, while internal nodes represent clusters formed through either merging or splitting of points. The dendrogram offers a multi-resolution view, enabling users to explore and select clusters at varying degrees of granularity. Unlike flat clustering methods (e.g., \(k\)-means), hierarchical methods do not require pre-specification of the number of clusters, providing a flexible structure adaptable to different analytical needs.

Hierarchical clustering methods are broadly categorized into two complementary approaches: agglomerative (bottom-up) and divisive (top-down).

\paragraph{Agglomerative Clustering (Bottom-Up Approach).}
Agglomerative hierarchical clustering (AHC) starts by placing each data point into its own singleton cluster. It then iteratively merges pairs of clusters that are closest according to a defined linkage criterion until a single encompassing cluster is obtained. The linkage criterion, which determines how inter-cluster distances are measured, greatly influences the shape and composition of resulting clusters. Common linkage methods include: [SCIPY DOCS HAVE CITS]

\begin{itemize}
    \item \textbf{Single-Linkage (Nearest-Neighbor):} Defines cluster proximity by the shortest distance between any two points from different clusters. While effective at capturing clusters with irregular shapes, single-linkage can be susceptible to chaining effects, producing loosely connected clusters.
    \item \textbf{Complete-Linkage (Farthest-Neighbor):} Defines cluster proximity as the greatest distance between points across clusters, favoring more compact and spherical clusters, and being robust to noise.
    \item \textbf{Average-Linkage (UPGMA):} Measures the average pairwise distance across all pairs of points from two clusters. This balanced approach avoids the extreme behaviors of single- and complete-linkage and is widely applied in practice.
    \item \textbf{Centroid-Linkage:} Uses the Euclidean distance between the centroids (means) of clusters. Centroid-linkage typically forms spherical clusters and can be computationally efficient.
    \item \textbf{Ward’s Method:} Minimizes within-cluster variance at each merge, producing highly compact and balanced clusters.
\end{itemize}

The agglomerative approach's primary advantage lies in its conceptual simplicity, interpretability, and deterministic nature, but it typically incurs a computational complexity of \(O(n^2\log n)\) or \(O(n^3)\), depending on the linkage method and implementation.

\paragraph{Divisive Clustering (Top-Down Approach).}
Divisive hierarchical clustering operates in the reverse direction: it begins by placing all data points into a single comprehensive cluster. At each iteration, this cluster is recursively split into smaller clusters according to a specified criterion—often focusing on maximizing inter-cluster distance or minimizing intra-cluster similarity—until each cluster contains only one data point. 

Divisive clustering provides a complementary perspective to agglomerative methods, allowing potentially clearer initial partitioning of the data. However, divisive algorithms generally require greater computational resources, as optimal splitting is typically more computationally intensive than merging clusters. Due to this increased complexity, divisive clustering is less commonly used in practice but is valuable in contexts where meaningful top-level partitions are prioritized.


\subsection{Cost Functions in Hierarchical Clustering}\label{subsec:hierarchical_clustering_cost}

To quantitatively assess and optimize hierarchical clustering, researchers have proposed explicit cost functions. Dasgupta introduced a formal, pairwise-similarity-based objective that allows rigorous evaluation and comparison of hierarchical clustering outcomes~\cite{dasgupta2016cost}. Specifically, given a weighted similarity graph \( G=(V,E,w) \), where each vertex represents a data point and each weighted edge \( w_{ij} \) quantifies the similarity between points \( i \) and \( j \), Dasgupta’s cost function for a hierarchical tree \( T \) is defined as:
\[
\text{cost}_G(T) = \sum_{\{i,j\}\in E} w_{ij}\,\big|\text{leaves}(T_{i \vee j})\big|,
\]
where \( \text{leaves}(T_{i \vee j}) \) is the set of leaves under the subtree rooted at the lowest common ancestor of \( i \) and \( j \). Intuitively, this cost penalizes hierarchies that delay clustering highly similar points, encouraging closely related items to cluster together early in the hierarchy.

The significance of Dasgupta’s cost function lies in its formalization of hierarchical clustering quality, enabling algorithmic comparisons through approximation ratios—how closely algorithms approximate the optimal hierarchy. Consequently, this cost function has prompted considerable theoretical research, revealing that while optimizing Dasgupta’s cost exactly is NP-hard, practical algorithms such as average-linkage agglomerative clustering achieve provable constant-factor approximations~\cite{charikar2019hierarchical}.

Recent studies have extended this framework, investigating alternative objectives (e.g., revenue maximization, robustness, interpretability) that prioritize different aspects of hierarchical clustering quality. However, even with explicit cost functions, certain intuitive attributes—like interpretability or alignment with external categories—might remain inadequately captured. [WHERE WAS THIS FROM UGH]


\subsection{Applications of Hierarchical Clustering.}
Hierarchical clustering's flexibility and interpretability have enabled it to become pervasive across diverse application areas. Notable examples include:

\begin{itemize}
    \item \textbf{Computational Biology and Bioinformatics:} Widely employed for gene expression analysis to discover clusters of co-expressed genes, thereby elucidating biological pathways and functions. Additionally, hierarchical clustering underpins the construction of phylogenetic trees, revealing evolutionary relationships among species or genetic sequences.
    
    \item \textbf{Image Processing and Computer Vision:} Hierarchical methods are integral to multi-scale image segmentation, effectively organizing pixels into coherent regions at various resolutions. Such techniques facilitate detailed scene understanding, object detection, and content-based image retrieval.
    
    \item \textbf{Natural Language Processing (NLP):} Extensively applied in document clustering to identify thematic groupings, thereby enabling structured information retrieval, summarization, and exploration of large textual corpora. Hierarchical clustering is also instrumental in developing taxonomies and concept hierarchies in ontology construction.
    
    \item \textbf{Marketing and Customer Analytics:} Businesses frequently utilize hierarchical clustering for market segmentation and customer profiling, revealing detailed consumer segments based on purchasing behaviors, demographic attributes, or online activities. This segmentation allows targeted marketing and personalized recommendations.
    
    \item \textbf{Social Network Analysis:} Hierarchical clustering aids in detecting community structures within networks, capturing meaningful groupings such as social circles, influencer communities, or collaborative groups. These structures can inform targeted interventions, marketing strategies, or community detection in online platforms.
    
    \item \textbf{Psychology and Sociology:} Researchers leverage hierarchical clustering for psychometric data analysis and social behavior pattern identification, uncovering latent constructs and behavioral archetypes within populations. This facilitates targeted policy-making, interventions, and sociocultural studies.
    
    \item \textbf{Healthcare and Medical Diagnostics:} Hierarchical methods help classify patients based on symptom profiles or diagnostic test results, improving patient stratification, treatment personalization, and clinical decision support systems.
\end{itemize}

The broad applicability across these domains underscores hierarchical clustering’s effectiveness in capturing complex structures inherent to diverse datasets, aligning closely with many practical, societal, and ethical contexts discussed earlier. Nonetheless, the computational complexity of hierarchical clustering poses a persistent challenge, especially with massive datasets.


\section{Fair Hierarchical
Clustering}\label{subsec:fair_hierarchical_clustering}

Given the widespread application of hierarchical clustering in sensitive domains such as healthcare, hiring, criminal justice, and social network analysis, ensuring fairness becomes not only ethically imperative but also practically critical. \emph{Fair hierarchical clustering (FHC)} integrates explicit fairness constraints into hierarchical clustering methods, thereby producing cluster hierarchies that respect fairness at every hierarchical level.

\subsubsection{Problem Formulation}

Formally, fair hierarchical clustering requires that fairness constraints are satisfied at every internal node of the clustering hierarchy, rather than solely at a final partition. Consider a dataset \( X \) partitioned into \( \lambda \) protected groups (e.g., gender, ethnicity). A hierarchical clustering \( T \) is considered \emph{fair} if, for every cluster \( C \) in the hierarchy (excluding singletons), the representation of each protected group within that cluster remains within predefined bounds~\cite{knittel2023generalized}. 

Specifically, given parameters \(\alpha_i\), \(\beta_i\) for each group \( i \in [\lambda] \), cluster \( C \) is fair if the proportion of elements belonging to group \( i \) lies within the range:
\[
\beta_i \leq \frac{|C_i|}{|C|} \leq \alpha_i
\]
where \(|C_i|\) denotes the number of points from group \( i \) within cluster \( C \). A common special case, known as \emph{proportional fairness}, occurs when the group proportions within each cluster precisely reflect their proportions in the overall dataset, typically allowing some small tolerance (slack). For example, if the dataset contains 30\% members of one protected group and 70\% of another, each cluster at every level of the hierarchy must approximately reflect this 30/70 ratio~\cite{knittel2023generalized}. Such recursive enforcement of fairness constraints significantly complicates the hierarchical clustering problem, as decisions at higher levels impose stringent constraints on lower-level cluster splits.

Beyond proportional fairness, researchers have introduced the concept of \emph{relative balance}, which refers to structural or size-based balance in the hierarchy itself. Relative balance ensures clusters at each split are not overly skewed in size, avoiding trivial fairness solutions (e.g., singletons or tiny clusters) and enhancing interpretability. Determining suitable notions of balance (size, depth, or distribution of protected groups) remains an active research question, as balance criteria often conflict with optimizing traditional clustering objectives like Dasgupta’s cost~\cite{knittel2023generalized}.

The integration of fairness into hierarchical clustering introduces considerable theoretical and computational challenges. Ensuring local fairness at each step can conflict with global optimization objectives, such as Dasgupta's cost, often necessitating higher-cost merges to maintain fairness. Furthermore, the combinatorial nature of the problem—searching through an exponentially large space of feasible hierarchies constrained by fairness—amplifies complexity. These difficulties require novel formulations and algorithms explicitly designed to balance fairness and clustering quality.

\emph{(Open question: How can fairness and cluster balance constraints be effectively combined, and what constitutes a suitable balance metric in hierarchical clustering?)}

\subsubsection{Algorithmic Approaches and Advances}

Fair hierarchical clustering is a relatively recent research area. Early pioneering work by Ahmadian et al.~\cite{ahmadian2020fairhc} introduced the first formal definition of fair hierarchical clustering, extending the concept of "fairlets" (small, inherently fair subsets of points) from flat clustering into hierarchical clustering. In their approach, the hierarchy is constructed through modified agglomerative or divisive algorithms that ensure fairness at each intermediate merge or split. While groundbreaking, these initial algorithms had significant limitations, including restrictive assumptions (e.g., two equally sized groups) and relatively weak theoretical guarantees, achieving a cost approximation factor of roughly \( O(n^{5/6}\cdot\text{polylog}(n)) \)~\cite{ahmadian2020fairhc}.

Subsequent research by Knittel et al.~\cite{knittel2023generalized} significantly advanced this area by proposing generalized reduction methods. These methods systematically convert any initial (possibly unfair) hierarchical clustering into a fair and balanced hierarchy with provably small increases in cost. Central to their approach is a collection of carefully defined \emph{tree operators}, such as:

\begin{itemize}
    \item \textbf{Subtree Swap:} Exchanges subtrees or nodes between clusters to improve fairness ratios.
    \item \textbf{Leaf Promotion:} Moves certain leaves higher in the hierarchy to correct fairness imbalances.
    \item \textbf{Bifurcation Adjustment:} Modifies splits to evenly distribute protected groups across branches.
    \item \textbf{Merge/Split Adjustments:} Refines cluster boundaries at different hierarchical levels to maintain fairness.
\end{itemize}

By judiciously applying and scheduling these operators, the generalized reduction approach maintains fairness recursively while closely controlling clustering cost. Remarkably, Knittel et al.’s algorithms provide cost approximation guarantees that are polylogarithmic in \( n \), achieving nearly exponential improvements over earlier methods. Furthermore, their framework accommodates multiple protected groups with arbitrary proportions, significantly generalizing earlier restrictive settings~\cite{knittel2023generalized}.

Current state-of-the-art algorithms for fair hierarchical clustering can thus yield solutions provably within polylogarithmic factors of optimal clustering cost, while ensuring fairness at every hierarchical node. Nonetheless, ongoing research aims to further narrow the approximation gaps and enhance practical usability, since current methods involve intricate multi-phase adjustments.

Empirical evidence from recent studies suggests that enforcing fairness constraints typically induces only modest increases in clustering cost, making fair hierarchical clustering viable for real-world use~\cite{ahmadian2020fairhc,knittel2023generalized}. Still, considerable opportunities remain for refining algorithms, improving computational efficiency, and tailoring these methods for domain-specific constraints.

\emph{(Open question: Can fair hierarchical clustering algorithms be made more computationally efficient and practically scalable, and what domain-specific adjustments could enhance their effectiveness?)}

\section{Theory–Practice Gap and Benchmarking in Algorithmic Research}
\label{sec:theory_practice_benchmarking}

In algorithmic research, a significant challenge is the persistent
gap between theoretical guarantees and practical performance,
commonly known as the \emph{theory–practice gap}. This gap emerges
when algorithms that perform well under theoretical
models fail to deliver expected outcomes in real-world applications,
or conversely, when methods that excel in practice lack robust
theoretical foundations. Understanding and addressing this disparity
is critical for bridging academic innovation and practical implementation.

\subsection{Understanding the Theory–Practice Gap}

The theory–practice gap describes scenarios where theoretical
properties—such as optimality, approximation guarantees, or
polynomial runtimes—do not align closely with empirical results.
While theoretical analyses often focus on worst-case scenarios,
real-world data typically exhibit patterns or structures that differ
significantly from these cases. For instance, the
simplex algorithm for linear programming has exponential complexity
in the worst case but is remarkably efficient on typical real-world
problems [cite]. Conversely, theoretically optimal polynomial-time
algorithms, such as certain interior-point methods, sometimes perform
worse in practice due to high constant factors or implementation
complexity. [cite]

Another example is in reinforcement learning (RL), where numerous
theoretically sound algorithms with convergence guarantees fail to
outperform simpler heuristic-based methods in practical tasks like
game-playing or robotic control. This phenomenon was notably
highlighted at a recent NeurIPS conferences [cite], where the RL community
acknowledged the pressing need to integrate practical insights into
theoretical frameworks. Such examples illustrate how real-world
complexity, such as stochasticity and function approximation,
challenges simplified theoretical models.

Deep learning further epitomizes this gap. Despite the overwhelming
success of deep neural networks in practical applications,
theoretical explanations of their effectiveness, such as
generalization in overparameterized models, remain incomplete. [cite]
Techniques developed empirically, including batch normalization and
specific optimization heuristics, have often preceded theoretical
understanding, underscoring how empirical experimentation sometimes
outpaces theory. [cite]

The existence of this gap has implications for both research and
practice. Practitioners may disregard algorithms that, despite their
theoretical strengths, prove impractical due to complexity or poor
empirical performance. Conversely, theoretically weaker but
practically effective methods gain prominence, occasionally leading
to skepticism toward theoretical results. Therefore, efforts to
bridge this gap by conducting systematic empirical experiments on
algorithms that perform well in theory but have not been put through
rigorous real-world testing remain essential.

\subsection{Empirical Evaluation Methodologies}

Benchmarking provides a systematic approach to evaluate algorithms,
identifying precisely where theoretical expectations diverge from
practical outcomes. Effective empirical evaluation involves careful
dataset selection, appropriate metrics, and standardized evaluation
procedures. Benchmarks such as the DIMACS challenges for graph
algorithms [cite] or platforms like Kaggle competitions in machine learning
exemplify rigorous empirical frameworks, allowing researchers to
measure real-world performance consistently.

For fair algorithms, especially in unsupervised tasks like
clustering, benchmarking must also consider fairness metrics that we
have discussed previously
alongside traditional performance indicators like runtime and
clustering cost. This multi-dimensional evaluation helps stakeholders
understand algorithm performance comprehensively.

Benchmarking also uncovers performance nuances often missed in
theoretical analyses. Two algorithms with identical theoretical
guarantees may differ significantly in practice due to differences in
implementation details or heuristic optimizations. Empirical
evaluations can thus highlight strengths and weaknesses that purely
theoretical analyses might overlook.

Nevertheless, empirical evaluations face several challenges,
including reproducibility, diversity of benchmark datasets, and
avoiding benchmark overfitting. For instance, if the chosen datasets
do not represent real-world conditions adequately, the results may
not generalize beyond the benchmark suite. Additionally, fairness
evaluations introduce complexities, such as acquiring datasets with
accurate protected attribute labels and defining meaningful fairness
metrics reflective of societal concerns. [cite how?]

\subsection{Benchmarking Systems and Requirements for Fair Algorithms}

Effective benchmarking systems share several common features: a
diverse collection of test cases, standardized evaluation processes,
and transparent reporting of results. Established platforms such as
OpenML and MLCommons exemplify these characteristics, providing
robust frameworks for algorithm comparisons.

However, benchmarking fairness introduces additional considerations.
A fairness-specific benchmarking system should incorporate metrics
explicitly designed to measure fairness, clearly documenting how
fairness constraints influence algorithm performance. Such a system
must balance multiple objectives—like fairness versus clustering
quality—and present results transparently. [cite how?]

Transparency and reproducibility become even more critical in fair
benchmarking due to ethical implications. Open-source
implementations, publicly accessible datasets, and well-documented
evaluation protocols are crucial for validating claims about fairness
improvements. Additionally, benchmarks must include algorithms unfair
approaches to contextualize fairness improvements
relative to standard methods.

Lastly, stress-testing algorithms with challenging or edge-case
scenarios provides deeper insights into algorithm
robustness and highlights areas requiring further theoretical or
empirical attention.

\subsection{Future Directions}
(TODO: unsure of section title, since this is what the rest of the
thesis will deal with)

Despite progress, numerous open questions remain. One critical area
involves developing standardized benchmarks specifically for fair
clustering and other fairness-aware algorithmic tasks. Currently, no
widely adopted benchmarks exist for systematically evaluating
fairness in clustering contexts, highlighting a clear gap in the
research infrastructure.

Addressing these challenges and questions will be vital for closing
the theory–practice gap, ultimately leading to algorithms that are
both theoretically sound and practically impactful, particularly in
contexts where fairness is paramount.


\chapter*{Conclusion}
\addcontentsline{toc}{chapter}{Conclusion}
\chaptermark{Conclusion}
\markboth{Conclusion}{Conclusion}
\setcounter{chapter}{4}
\setcounter{section}{0}

Here's a conclusion, demonstrating the use of all that manual
incrementing and table of contents adding that has to happen if you
use the starred form of the chapter command. The deal is, the chapter
command in \LaTeX\ does a lot of things: it increments the chapter
counter, it resets the section counter to zero, it puts the name of
the chapter into the table of contents and the running headers, and
probably some other stuff.

So, if you remove all that stuff because you don't like it to say
``Chapter 4: Conclusion'', then you have to manually add all the
things \LaTeX\ would normally do for you. Maybe someday we'll write a
new chapter macro that doesn't add ``Chapter X'' to the beginning of
every chapter title.

\section{More info}
And here's some other random info: the first paragraph after a
chapter title or section head \emph{shouldn't be} indented, because
indents are to tell the reader that you're starting a new paragraph.
Since that's obvious after a chapter or section title, proper
typesetting doesn't add an indent there.

%If you feel it necessary to include an appendix, it goes here.
\appendix
\chapter{The First Appendix}
\chapter{The Second Appendix, for Fun}

%This is where endnotes are supposed to go, if you have them.
%I have no idea how endnotes work with LaTeX.

\backmatter % backmatter makes the index and bibliography appear
% properly in the t.o.c...

% if you're using bibtex, the next line forces every entry in the
% bibtex file to be included
% in your bibliography, regardless of whether or not you've cited it
% in the thesis.
\nocite{*}

% Rename my bibliography to be called "Works Cited" and not
% "References" or ``Bibliography''
% \renewcommand{\bibname}{Works Cited}

%    \bibliographystyle{bsts/mla-good} % there are a variety of
% styles available;
%  \bibliographystyle{plainnat}
% replace ``plainnat'' with the style of choice. You can refer to
% files in the bsts or APA
% subfolder, e.g.
% \bibliographystyle{APA/apa-good}  % or
% \bibliography{thesis}
% Comment the above two lines and uncomment the next line to use
% biblatex-chicago.
\printbibliography[heading=bibintoc]

% Finally, an index would go here... but it is also optional.
\end{document}
