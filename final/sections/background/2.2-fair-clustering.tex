\subsection{Clustering Algorithms}
A clustering algorithm \(A\) partitions an input dataset \(X \in
\mathbb{R}^{n \times m}\) into \(k\) clusters, where \(k \leq n\).
Formally, the algorithm outputs a set \(C = \{C_1, C_2, \dots,
C_k\}\), with each cluster \(C_i \subseteq X\), such that each data
sample \(x \in X\) is assigned to at least one cluster. Depending on
cluster assignment strategies, clustering is broadly categorized into two types:

\begin{itemize}
  \item \textbf{Hard clustering:} Each data point \(x\) belongs to
    exactly one cluster.
  \item \textbf{Soft clustering:} Data points may belong partially or
    probabilistically to multiple clusters.
\end{itemize}

The number of clusters \(k\) can either be provided as an input
parameter or determined by the clustering method itself. For
instance, in \(k\)-means, \(k\) is predefined, whereas hierarchical
clustering outputs a dendrogram without a predetermined \(k\),
allowing users to choose the number of clusters later.

\subsection{Taxonomy of Clustering Methods}
A diverse set of clustering methodologies exists, each employing
distinct strategies and assumptions. Following the taxonomy presented
by Xu et al. \cite{XuSurvey}, we classify clustering algorithms as follows:

\begin{enumerate}

  \item \textbf{Center-Based Clustering:} These algorithms partition
    the data to minimize an error metric between samples and their
    cluster center. The canonical example is \(k\)-means, which
    minimizes the squared Euclidean distance.

  \item \textbf{Hierarchical Clustering:} This approach creates a
    binary tree (dendrogram) where the root node represents the
    entire dataset, leaf nodes represent individual samples, and
    intermediate nodes represent clusters. Hierarchical clustering
    methods are either agglomerative (bottom-up) or divisive
    (top-down). We will specifically take a closer look at
    Hierarchical clustering methods in the following sections.

  \item \textbf{Mixture Model-Based Clustering:} A probabilistic
    approach that assumes data points originate from a mixture of
    underlying distributions. Algorithms in this class, such as
    Gaussian Mixture Models (GMM), optimize parameters to best fit
    the data distribution, typically via Expectation Maximization.

  \item \textbf{Graph-Based Clustering:} Data is first represented as
    a graph, where vertices represent samples, and edges denote
    similarity or proximity between samples. Clustering is performed
    by partitioning the graph based on its spectral properties,
    commonly by using eigenvectors of the Laplacian matrix.
    Graph-based clustering is particularly advantageous in scenarios
    involving complex relational data or non-convex clusters.

  \item \textbf{Fuzzy Clustering:} Unlike traditional clustering,
    fuzzy clustering assigns samples a grade of membership to
    clusters. Fuzzy C-Means (FCM) is the archetypal algorithm, where
    data points have partial cluster memberships, reflecting
    uncertainty or gradual boundaries between clusters.

  \item \textbf{Combinatorial Search-Based Clustering:} Many
    clustering objectives are NP-hard, prompting approaches to
    reformulate the clustering problem as a combinatorial
    optimization task. Evolutionary algorithms, genetic algorithms,
    and simulated annealing are typical techniques used to explore
    solutions efficiently.

\end{enumerate}

\subsection{Fairness in Machine Learning}
As previously discussed in Section~\ref{subsec:algorithmic_bias},
machine learning algorithms can unintentionally perpetuate and
amplify societal biases embedded in historical data. These biases
pose significant ethical and practical challenges. Consequently,
considerable research effort has been dedicated to mitigating bias
and ensuring fairness in machine learning models.

Fairness interventions can be applied at different stages of the
machine learning pipeline. Specifically, fairness strategies are
commonly categorized as follows:

\begin{enumerate}
  \item \textbf{Pre-processing}: The original dataset is modified
    prior to training or clustering to ensure fair representation or
    remove biases. This approach directly addresses biases inherent
    in training datasets and biases due to missing or
    underrepresented data, as introduced in
    Section~\ref{subsec:algorithmic_bias}.

  \item \textbf{In-processing}: Fairness constraints are integrated
    directly into the algorithm's learning objective. In-processing
    approaches typically modify optimization criteria or integrate
    fairness considerations into model structures, explicitly
    counteracting optimization biases highlighted in the previous
    discussion of algorithmic bias.

  \item \textbf{Post-processing}: Model outputs or clusters undergo
    modification after the learning or clustering process to meet
    fairness constraints. However, for clustering tasks,
    post-processing methods are less common due to the absence of
    labeled validation sets, as clustering is inherently unsupervised.
\end{enumerate}

Defining and enforcing fairness in unsupervised settings, such as
clustering, is especially challenging. In clustering tasks, unlike
supervised learning, labels for data
samples are unavailable. Consequently, the same dataset \(X\) is used
for both training and evaluating the clustering outcomes. This
complicates the definition and enforcement of fairness, as
conventional fairness measures for supervised methods rely on labels
to evaluate biases or discrimination.

The absence of explicit labels complicates direct application of
standard supervised fairness metrics. Consequently, specialized
fairness notions and metrics tailored explicitly for clustering tasks
have emerged.

\subsection{Fairness Notions for Clustering}
Fairness notions in clustering can be classified into these four categories:

\begin{itemize}
  \item \textbf{Group-Level Fairness}: Inspired by the Disparate
    Impact doctrine, these notions aim to ensure no protected group
    (e.g., based on ethnicity or gender) is disproportionately disadvantaged.
  \item \textbf{Individual-Level Fairness}: Ensures similar
    individuals receive similar clustering outcomes, typically
    defined using a dissimilarity metric.
  \item \textbf{Algorithm Agnostic Notions}: Defined independently of
    the clustering algorithm and applicable universally across
    clustering methods.
  \item \textbf{Algorithm Specific Notions}: Tailored specifically
    for certain clustering objectives or algorithms, such as the
    social fairness cost for \(k\)-means clustering.
\end{itemize}

As we move forward, the following mathematical definitions for prominent
fairness notions in clustering will be important.
[\cite{ChhabraOverview}ChabbraIEEE]

\begin{itemize}

  \item \textbf{Balance:} A cluster solution has a high balance if
    each protected group is proportionally represented in each
    cluster. For \(m\) protected groups, define \(r\) as the
    proportion of samples belonging to protected group \(b\) in the
    dataset and \(r_a\) as the proportion of group \(b\) members in
    cluster \(a\). Then, balance is defined as:
    \[
      \text{Balance} = \min_{a\in [k], b\in [m]}
      \min\left\{\frac{r}{r_a}, \frac{r_a}{r}\right\}
    \]

    Balance takes values between 0 and 1, with higher values
    indicating greater fairness.

  \item \textbf{Bounded Representation:} Enforces constraints on the
    maximum (\(\alpha\)) and minimum (\(\beta\)) proportions of
    protected group members in clusters. For each cluster \(a\) and
    protected group \(b\), the constraint is:
    \[
      \beta \leq P_{a,b} \leq \alpha, \quad \forall a\in [k], b\in [m]
    \]

  \item \textbf{Max Fairness Cost (MFC):} Measures deviation from a
    user-specified ideal proportion \(I_b\) for each protected group
    \(b\). The proportion of group \(b\) points in cluster \(a\) is
    \(P_{a,b}\), and MFC is defined as:
    \[
      \text{MFC} = \max_{a\in [k]} \sum_{b\in [m]} |P_{a,b} - I_b|
    \]

  \item \textbf{Social Fairness Cost:} For a set of cluster centers
    \(U\), the clustering cost for group \(a\) is \(O(U, X_a) =
    \sum_{x \in X_a} \min_{u \in U}\|x - u\|^2\), and the social
    fairness cost is defined as:
    \[
      \text{Social Fairness Cost} = \max_{a \in [m]} \frac{O(U, X_a)}{|X_a|}
    \]

    The objective here is to minimize this cost to ensure equitable
    clustering quality across groups.

  \item \textbf{Distributional Individual Fairness:} Assumes an
    \(f\)-divergence \(H_f(V_x \| V_y)\) measuring statistical
    distance between output distributions \(V_x\) and \(V_y\) of
    samples \(x\) and \(y\). Given a fairness similarity measure
    \(F(x,y)\), fairness requires:
    \[
      H_f(V_x \| V_y) \leq F(x,y), \quad \forall x,y \in X \times X
    \]

  \item \textbf{Kleindessner et al.'s Individual Fairness:} Ensures
    for each sample \(x\), belonging to cluster \(C_a\), the average
    distance \(d\) to samples in its cluster is at most the average
    distance to samples in any other cluster \(C_b\):
    \[
      \frac{1}{|C_a| - 1}\sum_{z \in C_a} d(x,z) \leq
      \frac{1}{|C_b|}\sum_{z \in C_b} d(x,z), \quad \forall b \neq a
    \]

  \item \textbf{Entropy (used mainly in deep clustering models)}:
    Defined based on the representation of each protected group in
    clusters. Let \(N_{a,b}\) denote the set of samples belonging to
    both cluster \(a\) and protected group \(b\), and \(n_a\) the
    total samples in cluster \(a\). Then entropy is defined as:
    \[
      \text{Entropy} = -\sum_{a \in [k]} \frac{|N_{a,b}|}{n_a} \log
      \frac{|N_{a,b}|}{n_a}
    \]

    Higher entropy indicates higher fairness within clusters.

\end{itemize}

Balancing fairness constraints with clustering quality often presents
trade-offs, motivating methods that aim to find \emph{Pareto-optimal}
solutions that best balance these competing objectives~\cite{ChhabraOverview}.
